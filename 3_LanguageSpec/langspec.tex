\documentclass[12pt]{article}
% Last Revision:

%New notation: \hfill\triangle for end of theorem.

% Setup:
\usepackage{amssymb,latexsym,amsmath,amsthm,changepage}
\usepackage{hyperref}
\usepackage[margin=1in]{geometry}
\usepackage[usenames,dvipsnames]{color}
\hypersetup{
	urlcolor=blue,
	citecolor=green,
	linkcolor=OliveGreen,
	colorlinks=true
	}
\makeatletter
\renewcommand\@makefntext[1]{\leftskip=0em\hskip-0em\@makefnmark#1}
\makeatother
\newcommand{\cmnt}[1]{}

\newtheorem{theorem}{Theorem}
\newtheorem*{solution}{Solution}

\newenvironment{exercise}[2][Exercise]{\begin{trivlist}
\item[\hskip \labelsep {\bfseries #1}\hskip \labelsep {\bfseries #2.}]}{\end{trivlist}}
\newenvironment{problem}[2][Problem]{\begin{trivlist}
\item[\hskip \labelsep {\bfseries #1}\hskip \labelsep {\bfseries #2.}]}{\end{trivlist}}
\theoremstyle{remark}

\newcommand{\cmpl}{\phantom{ }^\mathsf{c}}
\newcommand{\setn}{_{k \in \mathbb{N}}}
\newcommand{\R}{\mathbb{R}}
\newcommand{\Z}{\mathbb{Z}}
\newcommand{\N}{\mathbb{N}}
\newcommand{\Q}{\mathbb{Q}}
\newcommand{\F}{\mathbb{F}}
\newcommand{\C}{\mathbb{C}}
\newcommand*\w[1]{\overrightarrow{#1}}
\newcommand{\st}{\text{ s.t. }}
\newcommand{\wind}{\indent\indent}
\newcommand{\rind}{\indent\indent\indent}
\newcommand{\enfa}{$\epsilon$-\text{NFA}}
\newcommand{\viset}{\{\w{v_1}, \ldots, \w{v_n}\}}
\newcommand{\wiset}{\{\w{w_1}, \ldots, \w{w_n}\}}
\newcommand*\tund[2]{\underbrace{#1}_\text{#2}}
\newcommand*\tover[2]{\overbrace{#1}_\text{#2}}
\newcommand{\IF}{\text{ if }}
\newcommand{\then}{\text{ then }}
\newcommand{\ow}{\text{ otherwise }}
\newcommand{\elab}{\textbf{\color{blue}{ Elaboration Needed. }}}
\newcommand{\gr}{\text{growth}}
\newcommand{\ttt}[1]{\texttt{#1}}

% Document
\begin{document}
\hfill John Loeber\\
\noindent{\Large PL Research: Language Specification} \hfill \today
\vspace{1.2pt}
\hrule
\vspace*{1.8mm}
$$$$
\noindent 
To avoid leaving the project nameless any further, I propose that the program (i.e. the software being developed) will be called \textbf{Embi} (as a working name). Let's recap the features of Embi:
\begin{enumerate}
\item Graphical interface (``GUI'') for manipulating visual elements on the canvas, like in Paint, Photoshop, etc.
\item Data panel that stores the \textbf{data}: all information on all visual elements on the canvas. May be manipulated by the user.
\item Code panel that allows the user to modify the data using a domain-specific language, \textbf{Embil}.
\item History panel (tabbed with code) that records, in Embil, all changes made to the program state.
\end{enumerate}

\noindent
The purpose of this document is to lay out a language specification for a toy version of Embil (i.e. one that I can implement within the next few weeks). I'll need some guidance from you with respect to the proper PL 

\end{document}